\documentclass[12pt]{article} 

\begin{document}

\title{Lab 3 Assignment [Ushna Ijaz- 2019-CE-39]}
\maketitle

\section{Question 4.3.1}
Show that the solution of $T(n) = T(n - 1) + n$ is $O(n^2)$\\ 

Answer: 

We know $T(n) \le cn^2$ for some value of c\\
$T(n) \le c(n-1)^2 + n$ \\
$T(n) \le cn^2 - 2cn + c + n $

Taking c = 1, we get\\
$ n^2 - 2n + 1 + n $ \\
$ n^2 - n + 1 \le n^2 $ for $n \ge 1$ \\
$c \ge 1$

\section{Question 4.3.2}
Question: Show that the solution of $T(n) = T(\lceil n / 2 \rceil) + 1$ is $O(\lg n)$ \\

Answer: 

$T(n) \le c\lg(n - 2)$ \\ 
$T(n) \le c\lg(\lfloor n / 2 \rfloor - 2) + 1$ \\
$T(n) = c\lg((n - 2)/2) + 1 $ \\ 
$T(n) = c\lg(n - 2) - c\lg2 + 1 $ \\ 
$T(n) \le c\lg(n - 2) - c\lg2 + 1 $ \\ 
for all values of $c \ge 1 $

\section{Question 4.3.3}
Question: We saw that the solution of $T(n) = 2T(\lfloor n / 2 \rfloor) + n$ is $O(n\lg n)$. Show that the solution of this recurrence is also $\Omega(n\lg n)$. Conclude that the solution is $\Theta(n\lg n)$.

Answer: 

Our first guess $T(n) \le cn\lg n $ \\ 
$T(n) \le 2c(\lfloor n / 2 ) \lg (\rfloor n / 2) + n$ \\
$T(n) \le cn \lg (n / 2) + n$\\
$T(n) \le cn \lg n - cn \lg 2 + n$ \\
$T(n) \le cn \lg n + (1 - c) n$ \\
$T(n) \le cn \lg n$ \\
for all values of $ c \ge \frac{1}{2} $

Next, $T(n) \ge c(n + 2)\lg(n + 2)$ \\ 

$T(n) \ge 2c(\lfloor n / 2 + 2 ) \lg (\rfloor n / 2 + 2) + n$ \\

$T(n) \ge 2c(n / 2 -1 + 2 ) \lg (n / 2 -1 + 2) + n$ \\

$T(n) \ge 2c(n + 2 / 2 ) \lg (n + 2 \ 2 ) + n$ \\

$T(n) = c(n+2) \lg (n + 2) - c (n + 2) \lg 2 + n$ \\

$T(n) = c(n+2) \lg (n + 2) + 1 (1 - c)n - 2c $ \\
$T(n) = c(n+2) \lg (n + 2)$ \\

$n \ge 2c/1-c$ for all values of $ 0 \le c \le 1$


\section{Question 4.3.7}
Question: Using the master method in Section 4.5, you can show that the solution to the recurrence $T(n) = 4T(n / 3) + n$ is $T(n) = \Theta(n^{\log_3 4})$. Show that a substitution proof with the assumption $T(n) \le cn^{\log_3 4}$ fails. Then show how to subtract off a lower-order term to make the substitution proof work \\

Answer: 

assume $T(n) \le cn^{\log_3 4}$ \\ 
guess $T(n) \le (cn^{\log_3 4 } ) - n$ \\

$T(n) \le 4((cn^{\log_3 4 } ) - n )+ n$   \\

$T(n) \le (cn^{\log_3 4 } - 4n + n) $ \\

$T(n) \le cn^{\log_3 4 }- n$ 

\section{Question 4.3.8}
Question: Using the master method in Section 4.5, you can show that the solution to the recurrence $T(n) = 4T(n / 2) + n^2$ is $T(n) = \Theta(n^2)$. Show that a substitution proof with the assumption $T(n) \le cn^2$ fails. Then show how to subtract off a lower-order term to make the substitution proof work. 

Answer: 

$T(n) \le 4c(n / 2)^2 + n^2$ \\ 

	 $\le cn^2 + n^2 $ \\
Guess: $T(n) \le 4c(n / 2)^2 - n/2 ) + n$\\

$T(n) \le cn^2 - n$
	 
	 
	 
\section{Question 4.3.9}
Question: Solve the recurrence $T(n) = 3T(\sqrt n) + \log n$ by making a change of variables. Your solution should be asymptotically tight. Do not worry about whether values are integral.	



\section{Question 4.4.2}
Question: Use a recursion tree to determine a good asymptotic upper bound on the recurrence $T(n) = T(n / 2) + n^2$. Use the substitution method to verify your answer. 

Answer: 
We know that there are $\lg{n}$ levels\\
$ Sum \sum_{i=0}^{\lg n-1} (1/4)^i n^2 +1  = O(n^2) 	$ \\
Substitution method: \\
guess: $T(n) \le cn^2$ \\
$ T(n) \le  c(n/2)^2 + n^2 $\\
$ T(n) \le  c(n^2)/4 + n^2 $\\
$ T(n) \le  (c/4+1)n^2 $\\
$ T(n) \le  cn^2 $\\
for all values of $c \ge 4/3$

\section{Question 4.4.3}
Question: Use a reccursion tree to determine a good asymptotic upper bound on the recurrence $T(n) = 2T(n - 1) + 1$. Use the substitution method to verify your answer.

Answer: 
Identifying the depth \\
depth: $\lg n $ \\
each level adds up and we get $n^2$ leaves

$ T(n) \le Sum \sum_{i=0}^{\lg n-1} ((2^i n) + (2 ^{1-i}))+ \Theta(n^2) 	$ \\


$ T(n)\le Sum \sum_{i=0}^{\lg n-1} (2^i n) + Sum \sum_{i=0}^{\lg n-1} (2 ^{1-i})+ \Theta(n^2) 	$ \\

$\frac{2^(\lg n)-1 }{1} + 2 Sum \sum_{i=0}^{\lg n-1} (1/2)^i + \Theta(n^2) $\\

$\Theta(n^2) + n + 3 $\\
$\Theta(n^2) $\\

Substitution method: \\
Guess: $T(n) \le cn^2 + n$ \\

$ T(n) \le 4c(n/2)^2 + 2n$\\


$ T(n) \le cn^2 + 2n$\\

$ T(n) =\Theta(n^2) $ \\


\section{Question 4.4.4}

Question: Use a reccursion tree to determine a good asymptotic upper bound on the recurrence $T(n) = 2T(n - 1) + 1$. Use the substitution method to verify your answer.

Answer: depth = n \\
each level being $2^i and 2^n$

$ T(n) = Sum \sum_{i=0}^{n-1} 2^i + \Theta(2^n) 	$ \\
Substitution method: \\
Guess: $ T(n) \le c2^n + n$\\
$ T(n) \le 2c2^n + (n-1) +1$ \\
$T(n) = O(2^n)$ 


\section{Question 4.4.5}

Question: Use a reccursion tree to determine a good asymptotic upper bound on the recurrence $T(n) = T(n - 1) + T(n / 2) + n$. Use the substitution method to verify your answer.

Guess: 
$ T(n) \le c2^n - 4n$ \\

$ T(n) \le c2^{n-1} - 4(n-1) + 2^{n/2}c - \frac{4n}{2} + n $\\
$ T(n) \le c(2^{n-1}+ 2^{n/2}) - 4n -1$\\
$ T(n) \le c(2^{n-1}+ 2^{n/2}) - 4n -1$\\ 
$ T(n) \le c(2^{n-1}) - 4n$ \\
$ T(n) \le c(2^n)-4n$\\
$T(n) = O(2^n)$
for all values of $n \ge 1/2$

Guess: 
$ T(n) \ge cn^2$ \\
$ T(n) \ge cn^n - 2cn +1 +cn^2/4 + n$ \\
$ T(n) \ge (2/5)cn^n +(1+2c)n +1 $ \\
$ T(n) \ge cn^n +(1+2)n + 1$ \\
$ T(n) \ge cn^2$
for all values of $c \ge 1/3 $\\
$T(n) = O(n^2)$ \\

\section{Question 4.4.6}
Question: Argue that the solution to the recurrence $T(n) = T(n / 3) + T(2n / 3) + cn$, where $c$ is a constant, is $\Omega(n\lg n)$ by appealing to the recurrsion tree.

Answer: 
We know that the cost at each level of the tree is cn. The problem has stated $T(n) = \Omega(n\lg n)$ so now we know for this we find out the lower bound. The shortest path has $log_{3}n $ levels . The solution of the recurrence is $cn\log_{3} n$ = $\Omega(n\lg n)$


\section{Question 4.4.7}
Question: Draw the recursion tree for $T(n) = 4T(\lfloor n / 2 \rfloor) + cn$, where $c$ is a constant, and provide a tight asymptotic bound on its solution. Verify your answer with the substitution method.

Answer: Tree drawn in word document. 

depth: $\lg n $ \\
each level adds up and we get $n^2$ leaves

$ T(n) \le Sum \sum_{i=0}^{\lg n-1} ((2^i cn)+ \Theta(n^2) 	$ \\


$ T(n)\le Sum \sum_{i=0}^{\lg n-1} (2^i) + \Theta(n^2) 	$ \\

$\frac{2^(\lg n)-1 }{1} + \Theta(n^2) $\\

$\Theta(n^2) $\\
Guess: $ T(n) \le cn^2 + 2cn $ \\
$ T(n) \le 4cn^2 + 2cn + cn$ \\
$ T(n) \le cn^2+2cn$ \\
Guess:  $ T(n) \ge cn^2 + 2cn $ \\
$ T(n) \ge 4cn^2 + 2cn + cn$ \\
$ T(n) \ge cn^2+2cn$ \\

\section{Question 4.5.1}
Question: Use the master method to give tight asymptotic bounds for the following recurrences:

a. $T(n) = 2T(n / 4) + 1$.

b. $T(n) = 2T(n / 4) + \sqrt n$.

c. $T(n) = 2T(n / 4) + n$.

d. $T(n) = 2T(n / 4) + n^2$.

Answer:\\

a. $\Theta(n^{\log_4 2}) = \Theta(\sqrt n)$.\\

b. $\Theta(n^{\log_4 2}\lg n) = \Theta(\sqrt n\lg n)$.\\

c. $\Theta(n)$.\\

d. $\Theta(n^2)$\\


\section{Question 4.5.2}

Question: Professor Caesar wishes to develop a matrix-multiplication algorithm that is asymptotically faster than Strassen's algorithm. His algorithm will use the divide-and-conquer method, dividing each matrix into pieces of size $n / 4 \times n / 4$, and the divide and combine steps together will take $\Theta(n^2)$ time. He needs to determine how many subproblems his algorithm has to create in order to beat Strassen's algorithm. If his algorithm creates a subproblems, then the recurrence for the running time $T(n)$ becomes $T(n) = aT(n / 4) + \Theta(n^2)$. What is the largest integer value of $a$ for which Professor Caesar's algorithm would be asymptotically faster than Strassen's algorithm?

Answer: 

We take the case where $a<16$ and by that we find can tell the algorithm which will be $\Theta(n^2)$.\\
If we take a=16, in this case our algorithm will be $\Theta(n^2\log{n})$\\
We know that the running time for Strassen's algorithm is $\Theta(n)^{\lg{7}})$ for master to be faster than Strassen, we take $log_{4} a \le \log_{2} 7$ from this we conclude $7^2$ which is equal to 49.\\
By case 1 of the master theorem, $T(n) = \Theta (n)^(\log_{4}(48)))$\\

Therefore, the largest integer value of $a$ is 48


\section{Question 4.5.3}

Question: Use the master method to show that the solution to the binary-search recurrence $T(n) = T(n / 2) + \Theta(1)$ is $T(n) = \Theta(\lg n)$.

Answer\\
a=1\\
b=2\\
$T(n)=\Theta(1)$ \\
$T(n) = \Theta(\lg n)$ \\
$T(n)=\Theta(n^{\lg 1})$\\


\section{Question 4.5.4}
Question: Can the master method be applied to the recurrence $T(n) = 4T(n / 2) + n^2\lg n$? Why or why not? Give an asymptotic upper bound for this recurrence.

Answer: The master method cannot be applied. It is because we can see that all the 3 cases, one being, taking $a = 4$, $b = 2$, we have $f(n) = n^2\lg n \ne O(n^{2 - \epsilon}) \ne \Omega(n^{2 - \epsilon})$.\\

Prove: \\
$T(n) \le 4T(n / 2) + n^2\lg n $\\
$4c(n / 2)^2\lg^2(n / 2) + n^2\lg n$\\
$ cn^2\lg(n / 2)\lg n - cn^2\lg(n / 2)\lg 2 + n^2\lg n$\\
$ cn^2\lg^2 n - cn^2\lg n\lg 2 - cn^2\lg(n / 2) + n^2\lg n$\\
$cn^2\lg^2 n + (1 - c)n^2\lg n - cn^2\lg(n / 2)(c > 1)$\\
$T(n)\le cn^2\lg^2 n - cn^2\lg(n / 2)$\\
$T(n)\le cn^2\lg^2 n$\\

\end{document} 

 